
\subsection{Logistic Regression with a Neural Network mindset}

Welcome to the first (required) programming exercise of the deep learning specialization. In this notebook you will build your first image recognition algorithm. You will build a cat classifier that recognizes cats with 70\% accuracy!


\begin{figure}[h]
\begin{center}
\includegraphics[width=0.5\textwidth]{course1/cat}
\end{center}
\caption{cat}
\label{fig:cat}
\end{figure}

As you keep learning new techniques you will increase it to 80+ \% accuracy on cat vs. non-cat datasets. By completing this assignment you will:
\begin{itemize}
\item  Work with logistic regression in a way that builds intuition relevant to neural networks.

\item  Learn how to minimize the cost function.

\item  Understand how derivatives of the cost are used to update parameters.
\end{itemize}

Take your time to complete this assignment and make sure you get the expected outputs when working through the different exercises. In some code blocks, you will find a "\#GRADED FUNCTION: functionName" comment. Please do not modify these comments. After you are done, submit your work and check your results. You need to score 70\% to pass. Good luck :) !


{\textbf {Instructions}}:
\begin{itemize}
\item  Do not use loops (for/while) in your code, unless the instructions explicitly ask you to do so.
\end{itemize}

{\textbf {You will learn to}}:
\begin{itemize}
\item Build the general architecture of a learning algorithm, including:
\begin{itemize}
\item Initializing parameters
\item Calculating the cost function and its gradient
\item Using an optimization algorithm (gradient descent)
\end{itemize}
\item Gather all three functions above into a main model function, in the right order.
\end{itemize}


\subsubsection{Packages}

First, let's run the cell below to import all the packages that you will need during this assignment. 
\begin{itemize}
\item \href{www.numpy.org}{numpy} is the fundamental package for scientific computing with Python.
\item \href{http://www.h5py.org}{h5py} is a common package to interact with a dataset that is stored on an H5 file.
\item \href{http://matplotlib.org}{matplotlib} is a famous library to plot graphs in Python.
\item \href{http://www.pythonware.com/products/pil/}{PIL} are used here to test your model with your own picture at the end.
\end{itemize}

\begin{mypython}
import numpy as np
import matplotlib.pyplot as plt
import h5py
import scipy
from PIL import Image
from scipy import ndimage
from lr_utils import load_dataset

#matplotlib inline
\end{mypython}





\subsubsection{Overview of the Problem set}

{\textbf {Problem Statement}}: You are given a dataset ("data.h5") containing:
\begin{itemize}
\item a training set of m\_train images labeled as cat (y=1) or non-cat (y=0)
\item a test set of m\_test images labeled as cat or non-cat
\item each image is of shape (num\_px, num\_px, 3) where 3 is for the 3 channels (RGB). Thus, each image is square (height = num\_px) and (width = num\_px).
\end{itemize}

You will build a simple image-recognition algorithm that can correctly classify pictures as cat or non-cat.

Let's get more familiar with the dataset. Load the data by running the following code.

\begin{minted}{python}
# Loading the data (cat/non-cat)
train_set_x_orig, train_set_y, test_set_x_orig, test_set_y, classes = load_dataset()
\end{minted}

We added "\_orig" at the end of image datasets (train and test) because we are going to preprocess them. After preprocessing, we will end up with train\_set\_x and test\_set\_x (the labels train\_set\_y and test\_set\_y don't need any preprocessing).\\
Each line of your train\_set\_x\_orig and test\_set\_x\_orig is an array representing an image. You can visualize an example by running the following code. Feel free also to change the index value and re-run to see other images.

\begin{minted}{python}
# Example of a picture
index = 25
plt.imshow(train_set_x_orig[index])
print ("y = " + str(train_set_y[:, index]) + ", it's a '" + classes[np.squeeze(train_set_y[:, index])].decode("utf-8") +  "' picture.")
\end{minted}

Many software bugs in deep learning come from having matrix/vector dimensions that don't fit. If you can keep your matrix/vector dimensions straight you will go a long way toward eliminating many bugs.

{\textbf {Exercise}: Find the values for:
\begin{itemize}
\item m\_train (number of training examples)
\item m\_test (number of test examples)
\item num\_px (= height = width of a training image)
\end{itemize}

Remember that train\_set\_x\_orig is a numpy-array of shape (m\_train, num\_px, num\_px, 3). For instance, you can access m\_train by writing train\_set\_x\_orig.shape[0].

\begin{minted}{python}
### START CODE HERE ### (≈ 3 lines of code)
m_train = train_set_x_orig.shape[0]
m_test = test_set_x_orig.shape[0]
num_px = train_set_x_orig.shape[1]
### END CODE HERE ###

print ("Number of training examples: m_train = " + str(m_train))
print ("Number of testing examples: m_test = " + str(m_test))
print ("Height/Width of each image: num_px = " + str(num_px))
print ("Each image is of size: (" + str(num_px) + ", " + str(num_px) + ", 3)")
print ("train_set_x shape: " + str(train_set_x_orig.shape))
print ("train_set_y shape: " + str(train_set_y.shape))
print ("test_set_x shape: " + str(test_set_x_orig.shape))
print ("test_set_y shape: " + str(test_set_y.shape))
\end{minted}

Output for m\_train, m\_test and num\_px:
\begin{minted}{python}
Number of training examples: m_train = 209
Number of testing examples: m_test = 50
Height/Width of each image: num_px = 64
Each image is of size: (64, 64, 3)
train_set_x shape: (209, 64, 64, 3)
train_set_y shape: (1, 209)
test_set_x shape: (50, 64, 64, 3)
test_set_y shape: (1, 50)
\end{minted}


For convenience, you should now reshape images of shape (num\_px, num\_px, 3) in a numpy-array of shape (num\_px*num\_px*3, 1). After this, our training (and test) dataset is a numpy-array where each column represents a flattened image. There should be m\_train (respectively m\_test) columns.

{\textbf {Exercise}: Reshape the training and test data sets so that images of size (num\_px, num\_px, 3) are flattened into single vectors of shape  (num\_px*num\_px*3, 1).


A trick when you want to flatten a matrix X of shape (a,b,c,d) to a matrix X\_flatten of shape (b * c *d, a) is to use:
\begin{minted}{python}
X_flatten = X.reshape(-1,X.shape[0]) 
\end{minted}

\begin{minted}{python}
# Reshape the training and test examples

### START CODE HERE ### (≈ 2 lines of code)
train_set_x_flatten = train_set_x_orig.reshape(train_set_x_orig.shape[0],-1).T
test_set_x_flatten = test_set_x_orig.reshape(test_set_x_orig.shape[0],-1).T
### END CODE HERE ###

print ("train_set_x_flatten shape: " + str(train_set_x_flatten.shape))
print ("train_set_y shape: " + str(train_set_y.shape))
print ("test_set_x_flatten shape: " + str(test_set_x_flatten.shape))
print ("test_set_y shape: " + str(test_set_y.shape))
print ("sanity check after reshaping: " + str(train_set_x_flatten[0:5,0]))
\end{minted}

Output for train\_set\_x\_flatten shape, test\_set\_x\_flatten shape:
\begin{minted}{python}
train_set_x_flatten shape: (12288, 209)
train_set_y shape: (1, 209)
test_set_x_flatten shape: (12288, 50)
test_set_y shape: (1, 50)
sanity check after reshaping: [17 31 56 22 33]
\end{minted}

To represent color images, the red, green and blue channels (RGB) must be specified for each pixel, and so the pixel value is actually a vector of three numbers ranging from 0 to 255.

One common preprocessing step in machine learning is to center and standardize your dataset, meaning that you substract the mean of the whole numpy array from each example, and then divide each example by the standard deviation of the whole numpy array. But for picture datasets, it is simpler and more convenient and works almost as well to just divide every row of the dataset by 255 (the maximum value of a pixel channel).

Let's standardize our dataset.

\begin{minted}{python}
train_set_x = train_set_x_flatten/255.
test_set_x = test_set_x_flatten/255.
\end{minted}

{\color{blue}
{\textbf {What you need to remember}}:
 
Common steps for pre-processing a new dataset are:
\begin{itemize}
\item Figure out the dimensions and shapes of the problem (m\_train, m\_test, num\_px, ...)
\item Reshape the datasets such that each example is now a vector of size (num\_px * num\_px * 3, 1)
\item "Standardize" the data
\end{itemize}
}




\subsubsection{General Architecture of the learning algorithm}

\indent It's time to design a simple algorithm to distinguish cat images from non-cat images.

You will build a Logistic Regression, using a Neural Network mindset. The following Figure explains why {\textbf {Logistic Regression is actually a very simple Neural Network}}!

\begin{figure}[h]
\begin{center}
\includegraphics[width=0.7\textwidth]{course1/LogReg}
\end{center}
\caption{Principle of Logistic Regression }
\label{fig:LogReg}
\end{figure}


{\textbf {Mathematical expression of the algorithm}}:

For one example $x^{(i)}$:
\begin{align}
z^{(i)} &= w^T x^{(i)} + b \\
\hat{y}^{(i)} &= a^{(i)} = sigmoid(z^{(i)})\\
\mathcal{L}(a^{(i)}, y^{(i)}) &=  - y^{(i)}  \log(a^{(i)}) - (1-y^{(i)} )  \log(1-a^{(i)})
\end{align}


The cost is then computed by summing over all training examples:
\begin{align}
J = \frac{1}{m} \sum_{i=1}^m \mathcal{L}(a^{(i)}, y^{(i)})
\end{align}

{\textbf {Key steps}}: In this exercise, you will carry out the following steps:
\begin{itemize}
\item Initialize the parameters of the model
\item Learn the parameters for the model by minimizing the cost  
\item Use the learned parameters to make predictions (on the test set)
\item Analyse the results and conclude
\end{itemize}



\subsubsection{Building the parts of our algorithm}

The main steps for building a Neural Network are:
\begin{itemize}
\item Define the model structure (such as number of input features)
\item Initialize the model's parameters
\item Loop:
\begin{itemize}
\item Calculate current loss (forward propagation)
\item Calculate current gradient (backward propagation)
\item Update parameters (gradient descent)
\end{itemize}
\end{itemize}
You often build 1-3 separately and integrate them into one function we call model().


\subsubsubsection{Helper functions}

{\textbf {Exercise}}: Using your code from "Python Basics", implement sigmoid(). As you've seen in the figure above, you need to compute $sigmoid( w^T x + b) = \frac{1}{1 + e^{-(w^T x + b)}}$ to make predictions. Use np.exp().

\begin{mypython}
# GRADED FUNCTION: sigmoid

def sigmoid(z):
    """
    Compute the sigmoid of z

    Arguments:
    z -- A scalar or numpy array of any size.

    Return:
    s -- sigmoid(z)
    """

    ### START CODE HERE ### (≈ 1 line of code)
    s = 1/(1+np.exp(-z))
    ### END CODE HERE ###
    
    return s
\end{mypython}



\subsubsubsection{Initializing parameters}

{\textbf {Exercise}}: Implement parameter initialization in the cell below. You have to initialize w as a vector of zeros. If you don't know what numpy function to use, look up np.zeros() in the Numpy library's documentation.
\begin{mypython}
def initialize_with_zeros(dim):
    """
    This function creates a vector of zeros of shape (dim, 1) for w and initializes b to 0.
    
    Argument:
    dim -- size of the w vector we want (or number of parameters in this case)
    
    Returns:
    w -- initialized vector of shape (dim, 1)
    b -- initialized scalar (corresponds to the bias)
    """
    
    ### START CODE HERE ### (≈ 1 line of code)
    w = np.zeros((dim,1))
    b = 0
    ### END CODE HERE ###

    assert(w.shape == (dim, 1))
    assert(isinstance(b, float) or isinstance(b, int))
    
    return w, b
\end{mypython}

For image inputs, w will be of shape (num\_px  $\times$  num\_px  $\times$  3, 1).


\subsubsubsection{Forward and Backward propagation}

Now that your parameters are initialized, you can do the "forward" and "backward" propagation steps for learning the parameters.

{\textbf {Exercise}}: Implement a function propagate() that computes the cost function and its gradient.

{\textbf {Hints}}:

Forward Propagation:
\begin{itemize}
\item You get X
\item You compute $A = \sigma(w^T X + b) = (a^{(0)}, a^{(1)}, ..., a^{(m-1)}, a^{(m)})$
\item You calculate the cost function: $J = -\frac{1}{m}\sum_{i=1}^{m}y^{(i)}\log(a^{(i)})+(1-y^{(i)})\log(1-a^{(i)})$
\end{itemize}

Here are the two formulas you will be using: 
\begin{align}
\frac{\partial J}{\partial w} &= \frac{1}{m}X(A-Y)^T\\
\frac{\partial J}{\partial b} &= \frac{1}{m} \sum_{i=1}^m (a^{(i)}-y^{(i)})
\end{align}

Code is as follows:
\clearpage
\begin{mypython}
# GRADED FUNCTION: propagate

def propagate(w, b, X, Y):
    """
    Implement the cost function and its gradient for the propagation explained above

    Arguments:
    w -- weights, a numpy array of size (num_px * num_px * 3, 1)
    b -- bias, a scalar
    X -- data of size (num_px * num_px * 3, number of examples)
    Y -- true "label" vector (containing 0 if non-cat, 1 if cat) of size (1, number of examples)

    Return:
    cost -- negative log-likelihood cost for logistic regression
    dw -- gradient of the loss with respect to w, thus same shape as w
    db -- gradient of the loss with respect to b, thus same shape as b
    
    Tips:
    - Write your code step by step for the propagation. np.log(), np.dot()
    """
    
    m = X.shape[1]
    
    # FORWARD PROPAGATION (FROM X TO COST)
    ### START CODE HERE ### (≈ 2 lines of code)
    A =  sigmoid(np.dot(w.T,X)+b)    # compute activation
    cost =  -(np.dot(Y,np.log(A.T))+np.dot(np.log(1-A),(1-Y).T))/m  # compute cost
    ### END CODE HERE ###
    
    # BACKWARD PROPAGATION (TO FIND GRAD)
    ### START CODE HERE ### (≈ 2 lines of code)
    dw = np.dot(X,(A-Y).T)/m
    db = np.sum(A-Y)/m
    ### END CODE HERE ###

    assert(dw.shape == w.shape)
    assert(db.dtype == float)
    cost = np.squeeze(cost)
    assert(cost.shape == ())
    
    grads = {"dw": dw,
             "db": db}
    
    return grads, cost
\end{mypython}



\subsubsubsection{Optimization}
\begin{itemize}
\item You have initialized your parameters.
\item You are also able to compute a cost function and its gradient.
\item Now, you want to update the parameters using gradient descent.
\end{itemize}

{\textbf {Exercise}}: Write down the optimization function. The goal is to learn $w$ and $b$ by minimizing the cost function $J$. For a parameter $\theta$, the update rule is $ \theta = \theta - \alpha \text{ } d\theta$, where $\alpha$ is the learning rate.

\begin{minted}{python}
# GRADED FUNCTION: optimize

def optimize(w, b, X, Y, num_iterations, learning_rate, print_cost = False):
    """
    This function optimizes w and b by running a gradient descent algorithm
    
    Arguments:
    w -- weights, a numpy array of size (num_px * num_px * 3, 1)
    b -- bias, a scalar
    X -- data of shape (num_px * num_px * 3, number of examples)
    Y -- true "label" vector (containing 0 if non-cat, 1 if cat), of shape (1, number of examples)
    num_iterations -- number of iterations of the optimization loop
    learning_rate -- learning rate of the gradient descent update rule
    print_cost -- True to print the loss every 100 steps
    
    Returns:
    params -- dictionary containing the weights w and bias b
    grads -- dictionary containing the gradients of the weights and bias with respect to the cost function
    costs -- list of all the costs computed during the optimization, this will be used to plot the learning curve.
    
    Tips:
    You basically need to write down two steps and iterate through them:
        1) Calculate the cost and the gradient for the current parameters. Use propagate().
        2) Update the parameters using gradient descent rule for w and b.
    """
    
    costs = []
    
    for i in range(num_iterations):
        
        
        # Cost and gradient calculation (≈ 1-4 lines of code)
        ### START CODE HERE ### 
        grads, cost = propagate(w, b, X, Y)
        ### END CODE HERE ###
        
        # Retrieve derivatives from grads
        dw = grads["dw"]
        db = grads["db"]
        
        # update rule (≈ 2 lines of code)
        ### START CODE HERE ###
        w = w-learning_rate*dw
        b = b-learning_rate*db
        ### END CODE HERE ###
        
        # Record the costs
        if i % 100 == 0:
            costs.append(cost)
        
        # Print the cost every 100 training examples
        if print_cost and i % 100 == 0:
            print ("Cost after iteration %i: %f" %(i, cost))
    
    params = {"w": w,
              "b": b}
    
    grads = {"dw": dw,
             "db": db}
    
    return params, grads, costs
\end{minted}


\begin{minted}{python}
w, b, X, Y = np.array([[1.],[2.]]), 2., np.array([[1.,2.,-1.],[3.,4.,-3.2]]), np.array([[1,0,1]])
params, grads, costs = optimize(w, b, X, Y, num_iterations= 100, learning_rate = 0.009, print_cost = False)
print ("w = " + str(params["w"]))
print ("b = " + str(params["b"]))
print ("dw = " + str(grads["dw"]))
print ("db = " + str(grads["db"]))

#output
w = [[ 0.19033591]
 [ 0.12259159]]
b = 1.92535983008
dw = [[ 0.67752042]
 [ 1.41625495]]
db = 0.219194504541
\end{minted}



{\textbf {Exercise}}: The previous function will output the learned w and b. We are able to use w and b to predict the labels for a dataset X. Implement the predict() function. There is two steps to computing predictions:
\begin{itemize}
\item Calculate $\hat{Y} = A = \sigma(w^T X + b)$

\item Convert the entries of a into 0 (if activation <= 0.5) or 1 (if activation > 0.5), stores the predictions in a vector `Y\_prediction'. If you wish, you can use an `if'/`else' statement in a `for' loop (though there is also a way to vectorize this). 
\end{itemize}

\begin{mypython}
def predict(w, b, X):
    '''
    Predict whether the label is 0 or 1 using learned logistic regression parameters (w, b)
    
    Arguments:
    w -- weights, a numpy array of size (num_px * num_px * 3, 1)
    b -- bias, a scalar
    X -- data of size (num_px * num_px * 3, number of examples)
    
    Returns:
    Y_prediction -- a numpy array (vector) containing all predictions (0/1) for the examples in X
    '''
    
    m = X.shape[1]
    Y_prediction = np.zeros((1,m))
    w = w.reshape(X.shape[0], 1)
    
    # Compute vector "A" predicting the probabilities of a cat being present in the picture
    ### START CODE HERE ### (≈ 1 line of code)
    A = sigmoid(np.dot(w.T,X)+b)
    ### END CODE HERE ###
    
    for i in range(A.shape[1]):
        
        # Convert probabilities A[0,i] to actual predictions p[0,i]
        ### START CODE HERE ### (≈ 4 lines of code)
        if A[0][i]<=0.5:A[0][i]=0
        else: A[0][i]=1
    Y_prediction=A
        ### END CODE HERE ###
    
    assert(Y_prediction.shape == (1, m))
    
    return Y_prediction
\end{mypython}

{\color{blue}\textbf{
What to remember: You've implemented several functions that:
\begin{itemize}
\item Initialize (w,b)
\item Optimize the loss iteratively to learn parameters (w,b):
\begin{itemize}
\item computing the cost and its gradient
\item updating the parameters using gradient descent
\end{itemize}
\item Use the learned (w,b) to predict the labels for a given set of examples
\end{itemize}
}}



\subsubsection{Merge all functions into a model}\label{sec:logistic_regression}

You will now see how the overall model is structured by putting together all the building blocks (functions implemented in the previous parts) together, in the right order.

{\textbf {Exercise}}: Implement the model function. Use the following notation:
\begin{itemize}
\item  Y\_prediction for your predictions on the test set
\item  Y\_prediction\_train for your predictions on the train set
\item  w, costs, grads for the outputs of optimize()
\end{itemize}

Code is as follows:
\begin{minted}{python}
# GRADED FUNCTION: model

def model(X_train, Y_train, X_test, Y_test, num_iterations = 2000, learning_rate = 0.5, print_cost = False):
    """
    Builds the logistic regression model by calling the function you've implemented previously
    
    Arguments:
    X_train -- training set represented by a numpy array of shape (num_px * num_px * 3, m_train)
    Y_train -- training labels represented by a numpy array (vector) of shape (1, m_train)
    X_test -- test set represented by a numpy array of shape (num_px * num_px * 3, m_test)
    Y_test -- test labels represented by a numpy array (vector) of shape (1, m_test)
    num_iterations -- hyperparameter representing the number of iterations to optimize the parameters
    learning_rate -- hyperparameter representing the learning rate used in the update rule of optimize()
    print_cost -- Set to true to print the cost every 100 iterations
    
    Returns:
    d -- dictionary containing information about the model.
    """
    
    ### START CODE HERE ###
    
    # initialize parameters with zeros (≈ 1 line of code)
    w, b = initialize_with_zeros(X_train.shape[0])

    # Gradient descent (≈ 1 line of code)
    parameters, grads, costs = optimize(w, b, X_train, Y_train, num_iterations, learning_rate, print_cost)
    
    # Retrieve parameters w and b from dictionary "parameters"
    w = parameters["w"]
    b = parameters["b"]
    
    # Predict test/train set examples (≈ 2 lines of code)
    Y_prediction_test = predict(w, b, X_test)
    Y_prediction_train = predict(w, b, X_train)

    ### END CODE HERE ###

    # Print train/test Errors
    print("train accuracy: {} %".format(100 - np.mean(np.abs(Y_prediction_train - Y_train)) * 100))
    print("test accuracy: {} %".format(100 - np.mean(np.abs(Y_prediction_test - Y_test)) * 100))

    
    d = {"costs": costs,
         "Y_prediction_test": Y_prediction_test, 
         "Y_prediction_train" : Y_prediction_train, 
         "w" : w, 
         "b" : b,
         "learning_rate" : learning_rate,
         "num_iterations": num_iterations}
    
    return d
\end{minted}   
    
    
Run the following cell to train your model.
\begin{minted}{python}
d = model(train_set_x, train_set_y, test_set_x, test_set_y, num_iterations = 2000, learning_rate = 0.005, print_cost = True)

#Output:

Cost after iteration 0: 0.693147
Cost after iteration 100: 0.584508
Cost after iteration 200: 0.466949
Cost after iteration 300: 0.376007
Cost after iteration 400: 0.331463
Cost after iteration 500: 0.303273
Cost after iteration 600: 0.279880
Cost after iteration 700: 0.260042
Cost after iteration 800: 0.242941
Cost after iteration 900: 0.228004
Cost after iteration 1000: 0.214820
Cost after iteration 1100: 0.203078
Cost after iteration 1200: 0.192544
Cost after iteration 1300: 0.183033
Cost after iteration 1400: 0.174399
Cost after iteration 1500: 0.166521
Cost after iteration 1600: 0.159305
Cost after iteration 1700: 0.152667
Cost after iteration 1800: 0.146542
Cost after iteration 1900: 0.140872
train accuracy: 99.04306220095694 %
test accuracy: 70.0 %
\end{minted}


{\textbf {Comment}}: Training accuracy is close to 100\%. This is a good sanity check: your model is working and has high enough capacity to fit the training data. Test error is 68\%. It is actually not bad for this simple model, given the small dataset we used and that logistic regression is a linear classifier. But no worries, you'll build an even better classifier next week!

Also, you see that the model is clearly overfitting the training data. Later in this specialization you will learn how to reduce overfitting, for example by using regularization. Using the code below (and changing the index variable) you can look at predictions on pictures of the test set.


Let's also plot the cost function and the gradients.
\begin{mypython}
# Plot learning curve (with costs)
costs = np.squeeze(d['costs'])
plt.plot(costs)
plt.ylabel('cost')
plt.xlabel('iterations (per hundreds)')
plt.title("Learning rate =" + str(d["learning_rate"]))
plt.show()
\end{mypython}

\begin{figure}[h]
\begin{center}
\includegraphics[width=0.7\textwidth]{course1/cost_curve}
\end{center}
\caption{cost function}
\label{fig:cost_function}
\end{figure}


{\textbf {Interpretation}}: You can see the cost decreasing. It shows that the parameters are being learned. However, you see that you could train the model even more on the training set. Try to increase the number of iterations in the cell above and rerun the cells. You might see that the training set accuracy goes up, but the test set accuracy goes down. This is called {\textbf {overfitting.}}

\begin{minted}{python}
d = model(train_set_x, train_set_y, test_set_x, test_set_y, num_iterations = 2500, learning_rate = 0.005, print_cost = True)

#Output:

train accuracy: 99.52153110047847 %
test accuracy: 68.0 %
\end{minted}



\subsubsection{Further analysis (optional/ungraded exercise)}

Congratulations on building your first image classification model. Let's analyze it further, and examine possible choices for the learning rate $\alpha$. 


{\textbf {Choice of learning rate}}

{\textbf {Reminder}}: In order for Gradient Descent to work you must choose the learning rate wisely. The learning rate $\alpha$ determines how rapidly we update the parameters. If the learning rate is too large we may "overshoot" the optimal value. Similarly, if it is too small we will need too many iterations to converge to the best values. That's why it is crucial to use a well-tuned learning rate.

Let's compare the learning curve of our model with several choices of learning rates. Run the cell below. This should take about 1 minute. Feel free also to try different values than the three we have initialized the learning\_rates variable to contain, and see what happens.

\begin{mypython}
learning_rates = [0.01, 0.001, 0.0001]
models = {}
for i in learning_rates:
    print ("learning rate is: " + str(i))
    models[str(i)] = model(train_set_x, train_set_y, test_set_x, test_set_y, num_iterations = 1500, learning_rate = i, print_cost = False)
    print ('\n' + "----------------------------------" + '\n')

for i in learning_rates:
    plt.plot(np.squeeze(models[str(i)]["costs"]), label= str(models[str(i)]["learning_rate"]))

plt.ylabel('cost')
plt.xlabel('iterations')

legend = plt.legend(loc='upper center', shadow=True)
frame = legend.get_frame()
frame.set_facecolor('0.90')
plt.show()
\end{mypython}


The result:
\begin{minted}{python}
learning rate is: 0.01
train accuracy: 99.52153110047847 %
test accuracy: 68.0 %

--------------------------------------

learning rate is: 0.001
train accuracy: 88.99521531100478 %
test accuracy: 64.0 %

--------------------------------------

learning rate is: 0.0001
train accuracy: 68.42105263157895 %
test accuracy: 36.0 %

--------------------------------------
\end{minted}

\begin{figure}[h]
\begin{center}
\includegraphics[width=0.7\textwidth]{course1/three_learning_rates}
\end{center}
\caption{compare the learning curve of model with three learning rates}
\label{three_learning_rates}
\end{figure}

{\color{red}\textbf {Interpretation}}:
\begin{itemize}
\item Different learning rates give different costs and thus different predictions results.
\item If the learning rate is too large (0.01), the cost may oscillate up and down. It may even diverge (though in this example, using 0.01 still eventually ends up at a good value for the cost).
\item A lower cost doesn't mean a better model. You have to check if there is possibly overfitting. It happens when the training accuracy is a lot higher than the test accuracy.
\item In deep learning, we usually recommend that you:
\begin{itemize}
\item Choose the learning rate that better minimizes the cost function.
\item If your model overfits, use other techniques to reduce overfitting. (We'll talk about this in later videos.)
\end{itemize}
\end{itemize}
}

\subsubsection{Test with your own image (optional/ungraded exercise)}
Congratulations on finishing this assignment. You can use your own image and see the output of your model. To do that:

1. Click on "File" in the upper bar of this notebook, then click "Open" to go on your Coursera Hub.

2. Add your image to this Jupyter Notebook's directory, in the "images" folder

3. Change your image's name in the following code

4. Run the code and check if the algorithm is right (1 = cat, 0 = non-cat)!


\begin{mypython}
my_image = "test2.jpg"   # change this to the name of your image file 

# We preprocess the image to fit your algorithm.
fname = "images/" + my_image
image = np.array(ndimage.imread(fname, flatten=False))
my_image = scipy.misc.imresize(image, size=(num_px,num_px)).reshape((1, num_px*num_px*3)).T
my_predicted_image = predict(d["w"], d["b"], my_image)

plt.imshow(image)
print("y = " + str(np.squeeze(my_predicted_image)) + ", your algorithm predicts a \"" + classes[int(np.squeeze(my_predicted_image)),].decode("utf-8") +  "\" picture.")
\end{mypython}


{\color{red}\textbf {What to remember from this assignment}}:

1. Preprocessing the dataset is important.

2. You implemented each function separately: initialize(), propagate(), optimize(). Then you built a model().

3. Tuning the learning rate (which is an example of a "hyperparameter") can make a big difference to the algorithm. You will see more examples of this later in this course!
}


\clearpage


\subsubsection{Code of Logistic Regression with a Neural Network}
\begin{minted}{python}

#Logistic Regression with a Neural Network mindset

# -*- coding: utf-8 -*-

import numpy as np
import matplotlib.pyplot as plt
import h5py
import scipy
from PIL import Image
from scipy import ndimage
from lr_utils import load_dataset

# Loading the data (cat/non-cat)
train_set_x_orig, train_set_y, test_set_x_orig, test_set_y, classes = load_dataset()


m_train = train_set_x_orig.shape[0]
m_test = test_set_x_orig.shape[0]
num_px = train_set_x_orig.shape[1]

# Reshape the training and test examples
train_set_x_flatten = train_set_x_orig.reshape(train_set_x_orig.shape[0],-1).T
test_set_x_flatten = test_set_x_orig.reshape(test_set_x_orig.shape[0],-1).T

#"Standardize" the data
train_set_x = train_set_x_flatten/255.
test_set_x = test_set_x_flatten/255.


# GRADED FUNCTION: sigmoid
def sigmoid(x):
    """
    Compute the sigmoid of x

    Arguments:
    x -- A scalar or numpy array of any size

    Return:
    s -- sigmoid(x)
    """

    s = 1/(1+np.exp(-x))
    
    return s


# GRADED FUNCTION: initialize_with_zeros
def initialize_with_zeros(dim):
    """
    This function creates a vector of zeros of shape (dim, 1) for w and initializes b to 0.
    
    Argument:
    dim -- size of the w vector we want (or number of parameters in this case)
    
    Returns:
    w -- initialized vector of shape (dim, 1)
    b -- initialized scalar (corresponds to the bias)
    """

    w = np.zeros((dim,1))
    b = 0

    assert(w.shape == (dim, 1))
    assert(isinstance(b, float) or isinstance(b, int))
    
    return w, b


# GRADED FUNCTION: propagate
def propagate(w, b, X, Y):
    """
    Implement the cost function and its gradient for the propagation explained above

    Arguments:
    w -- weights, a numpy array of size (num_px * num_px * 3, 1)
    b -- bias, a scalar
    X -- data of size (num_px * num_px * 3, number of examples)
    Y -- true "label" vector (containing 0 if non-cat, 1 if cat) of size (1, number of examples)

    Return:
    cost -- negative log-likelihood cost for logistic regression
    dw -- gradient of the loss with respect to w, thus same shape as w
    db -- gradient of the loss with respect to b, thus same shape as b
    
    Tips:
    - Write your code step by step for the propagation. np.log(), np.dot()
    """
    
    m = X.shape[1]
    
    # FORWARD PROPAGATION (FROM X TO COST)
    A =  sigmoid(np.dot(w.T,X)+b)    # compute activation
    cost =  -(np.dot(Y,np.log(A.T))+np.dot(np.log(1-A),(1-Y).T))/m  # compute cost
    
    # BACKWARD PROPAGATION (TO FIND GRAD)
    dw = np.dot(X,(A-Y).T)/m
    db = np.sum(A-Y)/m

    assert(dw.shape == w.shape)
    assert(db.dtype == float)
    cost = np.squeeze(cost)
    assert(cost.shape == ())
    
    grads = {"dw": dw,
             "db": db}
    
    return grads, cost

# GRADED FUNCTION: optimize
def optimize(w, b, X, Y, num_iterations, learning_rate, print_cost = False):
    """
    This function optimizes w and b by running a gradient descent algorithm
    
    Arguments:
    w -- weights, a numpy array of size (num_px * num_px * 3, 1)
    b -- bias, a scalar
    X -- data of shape (num_px * num_px * 3, number of examples)
    Y -- true "label" vector (containing 0 if non-cat, 1 if cat), of shape (1, number of examples)
    num_iterations -- number of iterations of the optimization loop
    learning_rate -- learning rate of the gradient descent update rule
    print_cost -- True to print the loss every 100 steps
    
    Returns:
    params -- dictionary containing the weights w and bias b
    grads -- dictionary containing the gradients of the weights and bias with respect to the cost function
    costs -- list of all the costs computed during the optimization, this will be used to plot the learning curve.
    
    Tips:
    You basically need to write down two steps and iterate through them:
        1) Calculate the cost and the gradient for the current parameters. Use propagate().
        2) Update the parameters using gradient descent rule for w and b.
    """
    
    costs = []
    
    for i in range(num_iterations):
        
        
        # Cost and gradient calculation (≈ 1-4 lines of code)
        grads, cost = propagate(w, b, X, Y)
        
        # Retrieve derivatives from grads
        dw = grads["dw"]
        db = grads["db"]
        
        # update rule
        w = w-learning_rate*dw
        b = b-learning_rate*db
        
        # Record the costs
        if i % 100 == 0:
            costs.append(cost)
        
        # Print the cost every 100 training examples
        if print_cost and i % 100 == 0:
            print ("Cost after iteration %i: %f" %(i, cost))
    
    params = {"w": w,
              "b": b}
    
    grads = {"dw": dw,
             "db": db}
    
    return params, grads, costs


# GRADED FUNCTION: predict
def predict(w, b, X):
    '''
    Predict whether the label is 0 or 1 using learned logistic regression parameters (w, b)
    
    Arguments:
    w -- weights, a numpy array of size (num_px * num_px * 3, 1)
    b -- bias, a scalar
    X -- data of size (num_px * num_px * 3, number of examples)
    
    Returns:
    Y_prediction -- a numpy array (vector) containing all predictions (0/1) for the examples in X
    '''
    
    m = X.shape[1]
    Y_prediction = np.zeros((1,m))
    w = w.reshape(X.shape[0], 1)
    
    # Compute vector "A" predicting the probabilities of a cat being present in the picture
    A = sigmoid(np.dot(w.T,X)+b)
    
    for i in range(A.shape[1]):
        
        # Convert probabilities A[0,i] to actual predictions p[0,i]
        if A[0][i]<=0.5:A[0][i]=0
        else: A[0][i]=1
    Y_prediction=A
    
    assert(Y_prediction.shape == (1, m))
    
    return Y_prediction

#==============================================================================
# Merge all functions into a model
#==============================================================================

def model(X_train, Y_train, X_test, Y_test, num_iterations = 2000, learning_rate = 0.5, print_cost = False):
    """
    Builds the logistic regression model by calling the function you've implemented previously
    
    Arguments:
    X_train -- training set represented by a numpy array of shape (num_px * num_px * 3, m_train)
    Y_train -- training labels represented by a numpy array (vector) of shape (1, m_train)
    X_test -- test set represented by a numpy array of shape (num_px * num_px * 3, m_test)
    Y_test -- test labels represented by a numpy array (vector) of shape (1, m_test)
    num_iterations -- hyperparameter representing the number of iterations to optimize the parameters
    learning_rate -- hyperparameter representing the learning rate used in the update rule of optimize()
    print_cost -- Set to true to print the cost every 100 iterations
    
    Returns:
    d -- dictionary containing information about the model.
    """
    
    # initialize parameters with zeros (≈ 1 line of code)
    w, b =  initialize_with_zeros(X_train.shape[0])

    # Gradient descent (≈ 1 line of code)
    parameters, grads, costs = optimize(w, b, X_train, Y_train, num_iterations, learning_rate, print_cost)
    
    # Retrieve parameters w and b from dictionary "parameters"
    w = parameters["w"]
    b = parameters["b"]
    
    # Predict test/train set examples (≈ 2 lines of code)
    Y_prediction_test = predict(w, b, X_test)
    Y_prediction_train = predict(w, b,X_train)

    # Print train/test Errors
    print("train accuracy: {} %".format(100 - np.mean(np.abs(Y_prediction_train - Y_train)) * 100))
    print("test accuracy: {} %".format(100 - np.mean(np.abs(Y_prediction_test - Y_test)) * 100))

    
    d = {"costs": costs,
         "Y_prediction_test": Y_prediction_test, 
         "Y_prediction_train" : Y_prediction_train, 
         "w" : w, 
         "b" : b,
         "learning_rate" : learning_rate,
         "num_iterations": num_iterations}
    
    return d


d = model(train_set_x, train_set_y, test_set_x, test_set_y, num_iterations = 2000, learning_rate = 0.005, print_cost = True)
\end{minted}

\clearpage